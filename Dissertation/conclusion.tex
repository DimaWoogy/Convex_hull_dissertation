\chapter*{Заключение}						% Заголовок
\addcontentsline{toc}{chapter}{Заключение}	% Добавляем его в оглавление

%% Согласно ГОСТ Р 7.0.11-2011:
%% 5.3.3 В заключении диссертации излагают итоги выполненного исследования, рекомендации, перспективы дальнейшей разработки темы.
%% 9.2.3 В заключении автореферата диссертации излагают итоги данного исследования, рекомендации и перспективы дальнейшей разработки темы.
%% Поэтому имеет смысл сделать эту часть общей и загрузить из одного файла в автореферат и в диссертацию:

Таким образом, в данной работе были получены следующие результаты:
\begin{enumerate}
	\item Проведённый анализ существующих алгоритмов построения выпуклой оболочки показал, что есть проблема. Используются только алгоритмы имеющие не оптимальную сложность из-за более быстрого времени работы за счёт маленькой константы.
	\item Был разработан новый алгоритм построения выпуклой оболочки конечного множества точек. Этот алгоритм отличается своей лучшей сложностью $O(n \log h)$ по сравнению с популярными использующимися аналогами. Также алгоритм имеет ряд дополнительных преимуществ, таких как вычисление приблизительного центра оболочки и возможность добавления точек после окончания работы.
	\item Была разработана методика сравнения алгоритмов, которая принимает в рассмотрение не только количество точек в изначальном множестве точек, но и количество точек, которые попадут в выпуклую оболочку.
	\item В результате сравнения было показано, что предлагаемый в работе алгоритм может быть использован и имеет преимущество по сравнению с существующим алгоритмом Грэхема на маленьком проценте точек, которые лежат в выпуклой оболочке.
\end{enumerate}

Результаты работы были представлены на конференции\cite{matrokhin2018convex}. Программный код, реализующий основные идеи, разработанные в данной работе, лежит в открытом доступе\cite{matrokhin2017github}.

Рекомендуется использовать разработанный алгоритм для построения выпуклой оболочки для алгоритмов компьютерного зрения при большом размере объектов и изображений. Необходимо заранее учитывать примерный процент точек, который будет находиться в выпуклой оболочке.

Разработанную методику сравнения алгоритмов рекомендуется использовать для предварительного сравнения алгоритмов перед использованием. После построения графиков и таблиц с помощью методики можно легко сделать выбор между алгоритмами в соответствии со своими практическими нуждами.

Данная работа может быть продолжена в дальнейшем в следующих направлениях:
\begin{enumerate}
	\item Использованное красно-чёрное дерево может быть не самым лучшим выбором. Существуют множество видов сбалансированных двоичных деревьев поиска: AVL-дерево, B-дерево, декартово дерево, список с пропусками, splay-дерево и так далее\cite{neerc2010algorithms}. Важным продолжением работы может быть исследование того, насколько хорошо работают те или иные виды деревьев применительно к алгоритму, предлагаемому в этой работе.
	\item В работе было показано, что алгоритм вычисляет не только выпуклую оболочку, но и её приблизительный центр. В дальнейшем необходимо исследовать насколько хорошо работает такой найденный центр применительно к вычислению дескриптора объекта на изображении.
	\item Было рассмотрено только построение выпуклой оболочки на плоскости. Дальнейшим улучшением может быть адаптация предлагаемых идей под проблему построение оболочки в трёхмерном пространстве. Чтобы продолжать работу в этом направлении необходимо рассмотреть такие структуры данных, как k-d дерево\cite{bentley1975multidimensional}.
	\item Начальный этап генерации точек на выпуклой оболочке в методике сравнения является достаточно теоретическим, точки редко выстраиваются в круг. Он может быть лучше приближен к практике с помощью некоего алгоритма генерации случайного выпуклого многоугольника.
	\item Разработанная методика сравнения может быть использована для сравнения существующих алгоритмов. Причём не только на плоскости, но и в n-мерном пространстве.
	\item Предлагаемый алгоритм построения выпуклых оболочек может быть расширен для построения вогнутых оболочек, что дает ещё больше пространства для применения его при вычисления дескрипторов объектов на изображениях~\cite{braune2016obtaining}.
\end{enumerate}
