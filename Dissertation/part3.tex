\chapter{Разработка предлагаемого алгоритма и системы сравнения быстродействия алгоритмов построения выпуклых оболочек} \label{chapt3}

\section{Разработка методики сравнения алгоритмов}

\subsection{Минусы классической методики сравнения}

Большинство сравнений алгоритмов построения выпуклой оболочки не зависят от выходных данных. А как было продемонстрировано в предыдущих главах именно от количества точек в финальной выпуклой оболочке зависит время работы многих из них.

Большинство сравнений полагается на некоторые допущения по случайному распределению точек на плоскости. Вот некоторые из них \cite{chadnov2004algorithmsComparison}:

\begin{enumerate}
	\item Равномерное распределение в единичном квадрате.
	\item Равномерное распределение в единичном круге.
	\item Нормальное распределение в единичном квадрате.
	\item Распределение Лапласа в единичном квадрате с центром распределения в точке $(0.5, 0.5)$.
	\item Равномерное распределение точек на окружности.
\end{enumerate}

%TODO: добавить рисунки разных распределений

Как видно из этого списка, количество точек в финальной выпуклой оболочке будет каким-то фиксированным для каждого из выбранного способа. Например, при распределении внутри единичного круга число вершин в выпуклой оболочке для $n$ точек будет $\theta(n^{1/3})$ \cite{algolist2010convexhull}. А для распределения на окружности очевидно, что количество точек в выпуклой оболочке будет равно изначальному количеству точек.

Такой подход не даёт полноту картины для разного количества точек в выпуклой оболочке. Он привязывает к сравнению всего лишь на основе одного параметра - количества точек. Поэтому было решено разработать новую методику сравнения алгоритмов, которая бы давала эту возможность.

\subsection{Идея сравнения на основе выходного параметра алгоритма}

Основная идея, которая будет использоваться при сравнении алгоритмов - это мы будем сравнивать не только опираясь на $n$ (изначальное количество точек), но и на $h$ (количество точек в выпуклой оболочке).

Для того, чтобы достичь этого, необходимо придумать способ генерировать тестовые данные с фиксированным процентом точек, которые будут в выпуклой оболочке. Сперва генерация происходит на окружности, эти точки точно будут на выпуклой оболочке. Это показано на рисунке \ref{img:points_gen_1}. После чего необходимо сгенерировать точки, которые будут лежать внутри выпуклой оболочки и не попадут в неё. Это делается с помощью генерации точек внутри круга, что показано на рисунке \ref{img:points_gen_2}. Финальным шагом мы перемешываем эти точки и всё. Входные данные с фиксированным процентом точек на выпуклой оболочке готовы.

\begin{figure}[H]
	{\centering
		\hfill
		\subbottom[\label{img:points_gen_1}]{%
			\includesvg[width=0.45\linewidth]{gen_1}}
		\hfill
		\subbottom[\label{img:points_gen_2}]{%
			\includesvg[width=0.45\linewidth]{gen_2}}
		\hfill
	}
	\caption{Генерирование точек с фиксированным процентом на выпуклой оболочке}
	\label{img:points_gen}
\end{figure}

\section{Сравнение алгоритмов}

\subsection{Описание используемых алгоритмов и других параметров сравнения}

\subsection{Результаты}

\section{Выводы}
