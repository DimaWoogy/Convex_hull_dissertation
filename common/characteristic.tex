
{\actuality} Вычислительная геометрия "--- это раздел информатики, который уже долгое время помогает решать задачи в самых разных областях прикладной деятельности. Научные исследования в этой области занимаются триангуляцией, определением пересечения или принадлежности объектов, построения выпуклой оболочки и т.д. В этой работе будут рассматриваться алгоритмы построения выпуклых оболочек. В компьютерном зрении выпуклые оболочки используются для распознавания образов и других анализов изображений с использованием особых точек. В компьютерной графике они могут использоваться для определения столкновения объектов. Построение выпуклой оболочки является неотъемлемой частью и многих других алгоритмов, например, построение диаграмм Вороного или триангуляции~\cite{deBerg2000ComputationalGeometry}.

Область вычислительной геометрии связанная с построением выпуклых оболочек имеет множество научных исследований. Огромное количество алгоритмов, приложений и методик было разработано на основе этой задачи. Однако это не означает, что существующие алгоритмы идеальны. Скорость работы "--- это одна из важнейших характеристик алгоритма, особенно для алгоритмов компьютерного зрения, в частности нахождения дескриптора объекта на изображении. Эта работа прежде всего была произведена, чтобы улучшить эту характеристику. Алгоритмы, которые имеют оптимальную сложность $O(n \log{h})$, имеют слишком большую константу, поэтому не используются на практике. Во всех библиотеках компьютерного зрения используются алгоритмы со сложностью $O(n \log{n})$, где $n$ - это количество точек в изначальном множестве точек, а $h$ - количество точек в итоговой выпуклой оболочке. Очевидно, что такие алгоритмы будут проигрывать в скорости работы на малом количестве точек в выпуклой оболочке.

{\object} выступают дескрипторы объектов на изображении на основе выпуклых оболочек.

{\subject} являются алгоритмы построения выпуклых оболочек.

{\aim} данной работы является увеличение скорости вычисления дескрипторов объектов на изображении и построения выпуклых оболочек.

Для~достижения поставленной цели необходимо было решить следующие {\tasks}:
\begin{enumerate}
	\item Аналитический обзор существующих алгоритмов.
	\item Формализация задачи построения выпуклой оболочки.
	\item Разработка нового алгоритма построения выпуклой оболочки, для его применения при вычисления дескриптора объекта на изображении.
	\item Разработка методики оценки быстродействия алгоритмов, которая учитывает не только количество точек в изначальном множестве.
	\item Программная реализация разработанной методики и алгоритма.
	\item Оценка быстродействия предлагаемого алгоритма с помощью разработанной методики.
\end{enumerate}

{\novelty}
\begin{enumerate}
	\item Предлагаемый алгоритм построения выпуклой оболочки работает лучше, чем существующие алгоритмы для среднего количества точек и малого процента точек в выпуклой оболочке.
	\item Разработанный алгоритм имеет среднюю сложность лучше, чем большинство популярных алгоритмов.
	\item Впервые была разработана методика оценки быстродействия алгоритмов построения выпуклых оболочек, основанная на размере выходных данных.
\end{enumerate}

{\influence} данной работы состоит в возможности применения нового алгоритма в системах компьютерного зрения, использующих вычисление дескрипторов. Например, распознавание лиц, достопримечательностей, текста и любых других объектов. Также важно само сравнение алгоритмов, которое проведено в третьей главе, потому что оно позволяет анализировать и выбирать алгоритм построения выпуклой оболочки для конкретного применения.

{\defpositions}
\begin{enumerate}
	\item Формализованное представление задачи построения выпуклой оболочки.
 	\item Более эффективный алгоритм построения выпуклой оболочки при вычислении дескрипторов объекта на изображении.
	\item Доказательство корректности работы алгоритма.
	\item Вычисление сложности алгоритма.
	\item Методика сравнения быстродействия алгоритмов построения выпуклых оболочек.
	\item Оценка быстродействия разработанного алгоритма с помощью предлагаемой методики.
\end{enumerate}

{\probation}
Основные результаты работы докладывались~на конференции 2018 IEEE Conference of Russian~\cite{matrokhin2018convex}. Исходный код, который реализует предлагаемую функциональность опубликован на github~\cite{matrokhin2017github}.
