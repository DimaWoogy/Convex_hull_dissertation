
{\actuality} Вычислительная геометрия - это раздел информатики, который уже долгое время помогает решать задачи в самых разных областях прикладной деятельности. Научные исследования в этой области занимаются триангуляцией, определением пересечения или принадлежности объектов, построения выклой оболочки и т.д. В этой работе будут рассматриваться алгоритмы построения выпуклых оболочек. В компьютерном зрении выпуклые оболочки используются для распознавания образов и других анализов изображений с использованием особых точек. В компьютерной графике они могут использоваться для определения столкновения объектов. В других алгоритмах вычислительной геометрии построение выпуклой оболочки является их неотъемлемой частью, например, построение диаграмм Вороного или построение триангуляции \cite{deBerg2000ComputationalGeometry}.

Область вычислительной геометрии связанная с построением выпуклых оболочек имеет множество научных исследований. Огромное количество алгоритмов, приложений и методик было разработано на основе этой задачи. Однако это не означает, что существующие алгоритмы идеальны. Скорость работы - это одна из важнейших характеристик алгоритма, особенно для алгоритов компьютерного зрения, в частности нахождения дескриптора объекта на изображении. Именно на улучшение этой характеристики направлена эта работа. Большинство популярных алгоритмов имеет сложность $O(n \log n)$, а алгоритмы, имеющие сложность $O(n \log h)$ имеют высокую константу, что замедляет время работы программ.

{\object} выступают дескрипторы объектов на изображении на основе выпуклых оболочек.

{\subject} являются алгоритмы построения выпуклых оболочек.

{\aim} данной работы является увеличение скорости работы вычисления дескрипторов объектов на изображении и построения выпуклых оболочек.

Для~достижения поставленной цели необходимо было решить следующие {\tasks}:
\begin{enumerate}[label=\arabic*)]
  \item проанализировать существующие алгоритмы;
  \item разработать методику сравнения скорости работы алгоритмов;
  \item разработать новый алгоритм построения выпуклой оболочки для вычисления дескриптора объекта на изображении;
  \item сравнить время работы алгоритмов с помощью разработанной методики;
\end{enumerate}

{\novelty}
\begin{enumerate}
  \item Впервые была разработана методика сравнения алгоритмов построения выпуклых оболочек, основанная на размере выходных данных.
  \item Было выполнено оригинальное исследование алгоритмов построения выпуклых оболочек.
  \item Был разработан алгоритм построения выпуклой оболочки, который в среднем случае работает лучше, чем популярные уществующие.
\end{enumerate}

{\influence} данной работы состоит в возможности применения нового алгоритма в системах компьютерного зрения, использующих вычисление дескрипторов. Например, распознавание лиц, достопримечательностей, текста и любых других объектов.

{\defpositions}
\begin{enumerate}
  \item Алгоритм построения выпуклой оболочки для применения его при вычислении дескрипторов объекта на изображении.
  \item Доказательство корректности работы алгоритма.
  \item Вычисление сложности алгоритма.
  \item Методика сравнения времени работы алгоритмов построения выпуклых оболочек.
  \item Подтверждение результатов исследования с помощью разработанной методики.
\end{enumerate}

